\chapter{Result Recording and Analysis}
\label{cha:ana-sim}

\section{Result Recording}
\label{sec:ana-sim:result-recording}

{\opp} provides built-in support for recording simulation results, via
\textit{output vectors} and \textit{output scalars}. Output vectors are
time series data, recorded from simple modules or channels. You can use
output vectors to record end-to-end delays or round trip times of packets,
queue lengths, queueing times, module state, link utilization, packet
drops, etc. -- anything that is useful to get a full picture of what
happened in the model during the simulation run.

Output scalars are summary results, computed during the simulation and
written out when the simulation completes. A scalar result may be an
(integer or real) number, or may be a statistical summary comprised of
several fields such as count, mean, standard deviation, sum, minimum,
maximum, etc., and optionally histogram data.

Results may be collected and recorded in two ways:

\begin{enumerate}
  \item Based on the signal mechanism, using declared statistics;
  \item Directly from C++ code, using the simulation library
\end{enumerate}

The second method has been the traditional way of recording results. The
first method, based on signals and declared statistics, was introduced in
{\opp} 4.1, and it is preferable because it allows you to always record the
results in the form you need, without requiring heavy instrumentation or
continuous tweaking of the simulation model.

\subsection{Using Signals and Declared Statistics}
\label{sec:ana-sim:signals-and-statistics}

This approach combines the signal mechanism (see
\ref{sec:simple-modules:signals}) and NED properties (see
\ref{sec:ned-lang:properties}) in order to de-couple the generation of
results from their recording, thereby providing more flexibility in what to
record and in which form. The details of the solution have been described
in section \ref{sec:simple-modules:signal-based-statistics} in detail; here we
just give a short overview.

Statistics are declared in the NED files with the \ttt{@statistic} property,
and modules emit values using the signal mechanism. The simulation framework
records data by adding special result file writer listeners to the signals.
By being able to choose what listeners to add, the user can control what to
record in the result files and what computations to apply before recording.
The aforementioned section \ref{sec:simple-modules:signal-based-statistics}
also explains how to instrument simple modules and channels for signals-based
result recording.

The signals approach allows for calculation of aggregate statistics (such as the
total number of packet drops in the network) and for implementing a warm-up
period without support from module code. It also allows you to write
dedicated statistics collection modules for the simulation, also without
touching existing modules.

The same configuration options that were used to control result recording
with \cclass{cOutVector} and \ffunc{recordScalar()} also work when utilizing
the signals approach, and there are extra configuration options to make
the additional possibilities accessible.

\subsection{Direct Result Recording}
\label{sec:ana-sim:direct-result-recording}

With this approach, scalar and statistics results are collected in class
variables inside modules, then recorded in the finalization phase via
\ffunc{recordScalar()} calls. Vectors are recorded using
\cclass{cOutVector} objects. Use \cclass{cStdDev} to record summary statistics
like mean, standard deviation, minimum/maximum, and histogram-like classes
(\cclass{cHistogram}, \cclass{cPSquare}, \cclass{cKSplit}) to record the
distribution. These classes are described in sections
\ref{sec:sim-lib:statistics} and \ref{sec:sim-lib:result-recording}.
Recording of individual vectors, scalars and statistics can be enabled or
disabled via the configuration (ini file), and it is also the place to set
up recording intervals for vectors.

The drawback of recording results directly from modules is that result
recording is hardcoded in modules, and even simple requirement changes
(e.g. record the average delay instead of each delay value, or vice versa)
requires either code change or an excessive amount of result collection
code in the modules.



\section{Configuring Result Collection}
\label{sec:ana-sim:config-results}

\subsection{Result File Names}
\label{sec:ana-sim:result-file-names}

Simulation results are recorded into \textit{output scalar files} that
actually hold statistics results as well, and \textit{output vector
files}. The usual file extension for scalar files is \ttt{.sca}, and
for vector files \ttt{.vec}.

Every simulation run generates a single scalar file and a vector file.
The file names can be controlled with the \fconfig{output-vector-file}
and \fconfig{output-scalar-file} options. These options rarely need
to be used, because the default values are usually fine. The defaults
are:

\begin{inifile}
output-vector-file = "${resultdir}/${configname}-${runnumber}.vec"
output-scalar-file = "${resultdir}/${configname}-${runnumber}.sca"
\end{inifile}

Here, \ttt{\$\{resultdir\}} is the value of the \fconfig{result-dir}
configuration option which defaults to \ttt{results/}, and
\ttt{\$\{configname\}} and \ttt{\$\{runnumber\}} are the name of
the configuration name in the ini file (e.g. \ttt{[Config PureAloha]}),
and the run number. Thus, the above defaults generate file names
like \ttt{results/PureAloha-0.vec}, \ttt{results/PureAloha-1.vec},
and so on.

\begin{note}
  In {\opp} 3.x, the default result file names were \ttt{omnetpp.vec} and
  \ttt{omnetpp.sca}, and scalar files were always appended to, rather than
  being overwritten as in the 4.x version. When needed, the old behavior
  for scalar files can be turned back on by setting
  \ttt{output-scalar-file-append=true} in the configuration.
\end{note}


\subsection{Enabling/Disabling Result Items}
\label{sec:ana-sim:disabling-result-items}

The recording of simulation results can be enabled/disabled at multiple levels
with various configuration options:

\begin{itemize}
  \item All recording from a \fprop{@statistic} can be enabled/disabled together
        using the \fconfig{statistic-recording} option;
  \item Recording of a scalar or a statistic object can be controlled with the
        \fconfig{scalar-recording} option;
  \item Recording of an output vector can be controlled with the \fconfig{vector-recording}
        option;
  \item Recording of the bins of a histogram object can be controlled with the
        \fconfig{bin-recording} option.
\end{itemize}

All the above options are boolean per-object options, thus, they have similar syntaxes:

\begin{itemize}
\item \textit{<module-path>.<statistic-name>.}\fconfig{statistic-recording}\ttt{ = true/false}
\item \textit{<module-path>.<scalar-name>.}\fconfig{scalar-recording}\ttt{ = true/false}
\item \textit{<module-path>.<vector-name>.}\fconfig{vector-recording}\ttt{ = true/false}
\item \textit{<module-path>.<histogram-name>.}\fconfig{bin-recording}\ttt{ = true/false}
\end{itemize}

For example, all recording from the following statistic

\begin{ned}
@statistic[queueLength](record=max,timeavg,vector);
\end{ned}

can disabled with this ini file line:

\begin{inifile}
**.queueLength.statistic-recording = false
\end{inifile}

When a scalar, vector, or histogram is recorded using a \fprop{@statistic},
its name is derived from the statistic name, by appending the recording
mode after a semicolon. For example, the above statistic will generate 	the
scalars named \ttt{queueLength:max} and \ttt{queueLength:timeavg}, and the
vector named \ttt{queueLength:vector}. Their recording can be individually
disabled with the following lines:

\begin{inifile}
**.queueLength:max.scalar-recording = false
**.queueLength:timeavg.scalar-recording = false
**.queueLength:vector.vector-recording = false
\end{inifile}

The statistic, scalar or vector name part in the key may also contain
wildcards. This can be used, for example, to handle result items with
similar names together, or, by using \ttt{*} as name, for filtering by
module or to disable all recording. The following example turns off
recording of all scalar results except those called \ttt{latency}, and those
produced by modules named \ttt{tcp}:

\begin{inifile}
**.tcp.*.scalar-recording = true
**.latency.scalar-recording = true
**.scalar-recording = false
\end{inifile}

To disable all result recording, use the following three lines:

\begin{inifile}
**.statistic-recording = false
**.scalar-recording = false
**.vector-recording = false
\end{inifile}

The first line is not strictly necessary. However, it may improve runtime
performance because it causes result recorders not to be added, instead of
adding and then disabling them.


\subsection{Selecting Recording Modes for Signal-Based Statistics}
\label{sec:ana-sim:configuring-recording-modes}

Signal-based statistics recording has been designed so that it can be
easily configured to record a ``default minimal'' set of results, a
``detailed'' set of results, and a custom set of results (by modifying
the previous ones, or defined from scratch).

Recording can be tuned with the \fconfig{result-recording-modes}
per-object configuration option. The ``object'' here is the statistic,
which is identified by the full path (hierarchical name) of the module or
connection channel object in question, plus the name of the statistic
(which is the ``index'' of \fprop{@statistic} property, i.e. the name in
the square brackets). Thus, configuration keys have the syntax
\textit{<module-path>.<statistic-name>.}\ttt{result-recording-modes=}.

The \fconfig{result-recording-modes} option accepts one or more items as value,
separated by comma. An item may be a result recording mode (surprise!), and
two words with a special meaning, \ttt{default} and \ttt{all}:

\begin{itemize}
\item A \textit{result recording mode} means any item that may occur in the
      \ttt{record} key of the \fprop{@statistic} property; for example,
      \ttt{count}, \ttt{sum}, \ttt{mean}, \ttt{vector((count-1)/2)}.
\item \tbf{\ttt{default}} stands for the set of non-optional items from
      the \fprop{@statistic} property's \ttt{record} list, that is, those
      without question marks.
\item \tbf{\ttt{all}} means all items from the \fprop{@statistic} property's
      \ttt{record} list, including the ones with question marks.
\end{itemize}

The default value is \ttt{default}.

A lone ``-'' as option value disables all recording modes.

\textit{Recording mode} items in the list may be prefixed with ``+'' or
``-'' to add/remove them from the set of result recording modes. The
initial set of result recording modes is \ttt{default}; if the first item
is prefixed with ``+'' or ``-'', then that and all subsequent items are
understood as modifying the set; if the first item does not start with with
``+'' or ``-'', then it replaces the set, and further items are understood
as modifying the set.

This sounds more complicated than it is; an example will make it clear.
Suppose we are configuring the following statistic:

\begin{ned}
@statistic[foo](record=count,mean,max?,vector?);
\end{ned}

With the following the ini file lines (see results in comments):

\begin{inifile}
**.result-recording-modes = default  # --> count, mean
**.result-recording-modes = all      # --> count, mean, max, vector
**.result-recording-modes = -        # --> none
**.result-recording-modes = mean     # --> only mean (disables 'default')
**.result-recording-modes = default,-vector,+histogram # --> count,mean,histogram
**.result-recording-modes = -vector,+histogram         # --> same as the previous
**.result-recording-modes = all,-vector,+histogram  # --> count,mean,max,histogram
\end{inifile}

Here is another example which shows how to write a more specific option
key. The following line applies to \ttt{queueLength} statistics of
\ttt{fifo[]} submodule vectors anywhere in the network:

\begin{inifile}
**.fifo[*].queueLength.result-recording-modes = +vector  # default plus vector
\end{inifile}

In the result file, the recorded scalars will be suffixed with the recording mode,
i.e. the mean of \ttt{queueingTime} will be recorded as \ttt{queueingTime:mean}.


\subsection{Warm-up Period}
\label{sec:ana-sim:warmup-period}

The \fconfig{warmup-period} option specifies the length of the initial
warm-up period. When set, results belonging to the first $x$ seconds
of the simulation will not be recorded into output vectors, and will
not be counted into the calculation of output scalars.
This option is useful for steady-state simulations. The default is 0s
(no warmup period).

Example:

\begin{inifile}
warmup-period = 20s
\end{inifile}

Results recorded via signal-based statistics automatically obey the warm-up
period setting, but modules that compute and record scalar results
manually (via \ffunc{recordScalar()}) need to be modified so that they take
the warm-up period into account.

\begin{note}
When configuring a warm-up period, make sure that modules that compute and
record scalar results manually via \ffunc{recordScalar()} actually obey the
warm-up period in the C++ code.
\end{note}

The warm-up period is available via the \ffunc{getWarmupPeriod()} method of
the simulation manager object, so the C++ code that updates the corresponding
state variables needs to be surrounded with an \textit{if} statement:

Old:

\begin{cpp}
dropCount++;
\end{cpp}

New:

\begin{cpp}
if (simTime() >= getSimulation()->getWarmupPeriod())
    dropCount++;
\end{cpp}


\subsection{Output Vectors Recording Intervals}
\label{sec:ana-sim:vector-recording-intervals}

The size of output vector files can easily reach several gigabytes,
but very often, only some of the recorded statistics are
interesting to the analyst. In addition to selecting which vectors to
record, {\opp} also allows one to specify one or more collection intervals.

The latter can be configured with the \fconfig{vector-recording-intervals}
per-object option. The syntax of the configuration option is
\textit{<module-path>.<vector-name>.}\ttt{vector-recording-intervals=}\textit{<intervals>},
where both \textit{<module-path>} and \textit{<vector-name>} may
contain wildcards (see \ref{sec:config-sim:wildcards}).
\textit{<vector-name>} is the vector name, or the name string of the
\ffunc{cOutVector} object. By default, all output vectors are turned
on for the whole duration the simulation.

One can specify one or more intervals in the \textit{<startTime>..<stopTime>}
syntax, separated by comma. \textit{<startTime>} or \textit{<stopTime>} need
to be given with measurement units, and both can be omitted to denote
the beginning and the end of the simulation, respectively.

The following example limits all vectors to three intervals, except
\ttt{dropCount} vectors which will be recorded during the whole
simulation run:

\begin{inifile}
**.dropCount.vector-recording-intervals = 0..
**.vector-recording-intervals = 0..1000s, 5000s..6000s, 9000s..
\end{inifile}

\subsection{Recording Event Numbers in Output Vectors}
\label{sec:ana-sim:vector-eventnum-recording}

A third per-vector configuration option is \fconfig{vector-record-eventnumbers},
which specifies whether to record event numbers for an output vector.
(Simulation time and value are always recorded. Event numbers are needed
by the Sequence Chart Tool, for example.) Event number recording is enabled
by default; it may be turned off to save disk space.

\begin{inifile}
**.vector-record-eventnumbers = false
\end{inifile}

If the (default) \cclass{cIndexedFileOutputVectorManager} class is used to
record output vectors, there are two more options to fine-tune its resource
usage. \ttt{output-vectors-memory-limit} specifies the total memory that
can be used for buffering output vectors. Larger values produce less
fragmented vector files (i.e. cause vector data to be grouped into larger
chunks), and therefore allow more efficient processing later.
\ttt{vector-max-buffered-values} specifies the maximum number of values to
buffer per vector, before writing out a block into the output vector file.
The default is no per-vector limit (i.e. only the total memory limit is in
effect.)


\subsection{Saving Parameters as Scalars}
\label{sec:ana-sim:saving-parameters-as-scalars}

When you are running several simulations with different parameter
settings, you'll usually want to refer to selected
input parameters in the result analysis as well -- for example when
drawing a throughput (or response time) versus load (or network
background traffic) plot. Average throughput or response time numbers
are saved into the output scalar files, and it is useful for the input
parameters to get saved into the same file as well.

For convenience, {\opp} automatically saves the iteration variables
into the output scalar file if they have numeric value, so they can
be referred to during result analysis.

\begin{warning}
    If an iteration variable has non-numeric value, it will not be recorded
    automatically and cannot be used during analysis. This can happen
    unintentionally if you specify units inside an iteration variable list:
\begin{inifile}
**.param = exponential( ${mean=0.2s, 0.4s, 0.6s} )  #WRONG!
**.param = exponential( ${mean=0.2, 0.4, 0.6}s )    #OK
\end{inifile}
\end{warning}

Module parameters can also be saved, but this has to be
requested by the user, by configuring \ttt{param-record-as-scalar=true} for the
parameters in question. The configuration key is a pattern that
identifies the parameter, plus \ttt{.param-record-as-scalar}. An example:

\begin{inifile}
**.host[*].networkLoad.param-record-as-scalar = true
\end{inifile}

This looks simple enough, however there are three pitfalls:
non-numeric parameters, too many matching parameters, and
random-valued volatile parameters.

First, the scalar file only holds numeric results, so non-numeric
parameters cannot be recorded -- that will result in a runtime
error.

Second, if wildcards in the pattern match too many parameters, that
might unnecessarily increase the size of the scalar file. For example,
if the \ttt{host[]} module vector size is 1000 in the example below, then the
same value (3) will be saved 1000 times into the scalar file, once for
each host.

\begin{inifile}
**.host[*].startTime = 3
**.host[*].startTime.param-record-as-scalar = true  # saves "3" once for each host
\end{inifile}

Third, recording a random-valued volatile parameter will just save a
random number from that distribution. This is rarely what you need, and
the simulation kernel will also issue a warning if this happens.

\begin{inifile}
**.interarrivalTime = exponential(1s)
**.interarrivalTime.param-record-as-scalar = true  # wrong: saves random values!
\end{inifile}

These pitfalls are quite common in practice, so it is usually better
to rely on the iteration variables in the result analysis.
That is, one can rewrite the above example as

\begin{inifile}
**.interarrivalTime = exponential( ${mean=1}s )
\end{inifile}

and refer to the \ttt{\$mean} iteration variable instead of the
interarrivalTime module parameter(s) during result analysis.
\ttt{param-record-as-scalar=true} is not needed, because iteration variables are
automatically saved into the result files.


\subsection{Recording Precision}
\label{sec:ana-sim:outputfile-precision}

Output scalar and output vector files are text files, and floating point
values (\ttt{double}s) are recorded into it using \ttt{fprintf()}'s
\ttt{"\%g"} format. The number of significant digits can be configured
using the \fconfig{output-scalar-precision} and \fconfig{output-vector-precision}
configuration options.

The default precision is 12 digits. The following has to be considered when
setting a different value:

IEEE-754 doubles are 64-bit numbers. The mantissa is 52 bits, which is
roughly equivalent to 16 decimal places (52*log(2)/log(10)). However, due
to rounding errors, usually only 12..14 digits are correct, and the rest is
pretty much random garbage which should be ignored. However, when you
convert the decimal representation back into a \ttt{double} for result
processing, an additional small error will occur, because 0.1, 0.01, etc.
cannot be accurately represented in binary. This conversion error is
usually smaller than what that the \ttt{double} variable already had
before recording into the file. However, if it is important, you can
eliminate this error by setting the recording precision to 16 digits or
more (but again, be aware that the last digits are garbage). The practical
upper limit is 17 digits, setting it higher doesn't make any difference in
\ttt{fprintf()}'s output.

% To see finite machine precision and rounding errors, try this code:
%
% \ begin{verbatim}
% double x = 0.1;
% while (true)  {
%    printf("%.15g\n", x);
%    x = x + 0.1;
% }
% \ end{verbatim}
%
% The following, more advanced version will also print the error of
% converting back from text to double:
%
% \ begin{verbatim}
% double x = 0.1;
% while (true) {
%     char line[120];
%     sprintf(line, "%.15g \t%.14g \t%.13g \t%.12g", x, x, x, x);
%     double x15, x14, x13, x12;
%     sscanf(line, "%lg%lg%lg%lg", &x15, &x14, &x13, &x12);
%     printf("%s \t| %g  %g  %g  %g\n", line, (x15-x), (x14-x), (x13-x), (x12-x));
%     x = x + 0.1;
% }
% \ end{verbatim}
%    s
% For the complexity of the issue, see "What Every Computer Scientist
% Should Know About Floating-Point Arithmetic" by David Goldberg.

Errors resulting from converting to/from decimal representation can be
eliminated by choosing an output vector/output scalar manager class
which stores \ttt{double}s in their native binary form.
The appropriate configuration options are \fconfig{outputvectormanager-class}
and \fconfig{outputvectormanager-class}. For example,
\cclass{cMySQLOutputScalarManager} and \cclass{cMySQLOutputScalarManager}
provided in \ttt{samples/database} fulfill this requirement.

However, before worrying too much about rounding and conversion errors,
consider the \textit{real} accuracy of your results:

\begin{itemize}
  \item In real life, it is very difficult to measure quantities (weight, distance,
     even time) with more than a few digits of precision. What precision
     are your input data? For example, if you approximate inter-arrival
     time as \textit{exponential(0.153)} when the mean is really
     \textit{0.152601...} and the distribution is not even exactly exponential,
     you are already starting out with a bigger error than rounding can cause.

  \item The simulation model is itself an approximation of real life. How much
     error do the (known and unknown) simplifications cause in the results?
\end{itemize}

%% TODO also hint that results can be directed to database etc! by changing the implementation that cEnvir methods delegate to. (list cEnvir methods!)


\section{Result Files}
\label{sec:ana-sim:result-files}

\subsection{The {\opp} Result File Format}
\label{sec:ana-sim:omnetpp-result-file-format}

By default, {\opp} saves simulation results into textual, line-oriented files.
The advantage of a text-based, line-oriented format is that it is very
accessible and easy to parse with a wide range of tools and languages, and
still provides enough flexibility to be able to represent the data it
needs to (in contrast to e.g. CSV). This section provides an overview of
these file formats (output vector and output scalar files); the precise
specification is available in the Appendix (\ref{cha:result-file-formats}).

%XXX move away:
%  \footnote{Recording is actually configurable, and one can record
%  results into a database as well, by writing appropriate result
%  manager classes and activating them in the configuration.}

By default, each file contains data from one run only.

Result files start with a header that contains several attributes of the
simulation run: a reasonably globally unique run ID, the network NED type
name, the experiment-measurement-replication labels, the values of
iteration variables and the repetition counter, the date and time, the host
name, the process id of the simulation, random number seeds, configuration
options, and so on. These data can be useful during result processing, and
increase the reproducibility of the results.

%%FIXME example header!

Vectors are recorded into a separate file for practical reasons: vector
data usually consume several magnitudes more disk space than scalars.


\subsubsection{Output Vector Files}
\label{sec:ana-sim:output-vector-files}

All output vectors from a simulation run are recorded into the same file.
The following sections describe the format of the file, and
how to process it.

An example file fragment (without header):

\begin{filelisting}
...
vector 1   net.host[12]  responseTime  TV
1  12.895  2355.66
1  14.126  4577.66664666
vector 2   net.router[9].ppp[0] queueLength  TV
2  16.960  2
1  23.086  2355.66666666
2  24.026  8
...
\end{filelisting}

There two types of lines: vector declaration lines (beginning with the word
\ttt{vector}), and data lines. A \textit{vector declaration line}
introduces a new output vector, and its columns are: vector Id, module of
creation, name of \cclass{cOutVector} object, and multiplicity (usually 1).
Actual data recorded in this vector are on \textit{data lines} which begin
with the vector Id. Further columns on data lines are the simulation time
and the recorded value.

% FIXME plus attribute lines!!! also event numbers

Since {\opp} 4.0, vector data are recorded into the file clustered by
output vectors, which, combined with index files, allows much more
efficient processing. Using the index file, tools can extract particular
vectors by reading only those parts of the file where the desired data are
located, and do not need to scan through the whole file linearly.


\subsubsection{Scalar Result Files}
\label{sec:ana-sim:scalar-result-files}

Fragment of an output scalar file (without header):

\begin{filelisting}
...
scalar "lan.hostA.mac" "frames sent"  99
scalar "lan.hostA.mac" "frames rcvd"  3088
scalar "lan.hostA.mac" "bytes sent"   64869
scalar "lan.hostA.mac" "bytes rcvd"   3529448
...
\end{filelisting}

Every scalar generates one \ttt{scalar} line in the file.

Statistics objects (\cclass{cStatistic} subclasses such as \cclass{cStdDev})
generate several lines: mean, standard deviation, etc.

%% FIXME TODO attributes, statistics example, etc


\subsection{SQLite Result Files}
\label{sec:ana-sim:sqlite-result-files}

Starting from version 5.1, {\opp} contains experimental support for
saving simulation results into SQLite database files. The perceived advantage
of SQLite is existing support in many existing tools and languages (no need to
write custom parsers), and being able to use the power of the SQL language
for queries. The latter is very useful for processing scalar results, and less
so for vectors and histograms.

To let a simulation record its results in SQLite format, add the following
configuration options to its \ffilename{omnetpp.ini}:

\begin{inifile}
outputvectormanager-class="omnetpp::envir::SqliteOutputVectorManager"
outputscalarmanager-class="omnetpp::envir::SqliteOutputScalarManager"
\end{inifile}

\begin{note}
Alternatively, to make SQLite the default format, recompile {\opp} with
\ttt{PREFER\_SQLITE\_RESULT\_FILES=yes} set in \ffilename{configure.user}.
(Don't forget to also run  \fprog{./configure} before \fprog{make}.)
\end{note}

The SQLite result files will be created with the same names as textual
result files. The two formats also store exactly the same data, only in
a different way (there is one-to-one correspondence between them.) The
Simulation IDE and \fprog{scavetool} also understand both formats.

\begin{hint}
If you want to get acquainted with the organization of SQLite result
files, exploring one in a graphical tool such as SQLiteBrowser or SQLite
Studio should be a good start.
\end{hint}

The database schema can be found in Appendix \ref{cha:result-file-formats}.

%TODO file size, performance


\subsection{Scavetool}
\label{sec:ana-sim:scavetool}
\index{scavetool}

{\opp}'s \fprog{opp\_scavetool} program is a command-line tool for exploring,
filtering and processing of result files, and exporting the result in formats
digestible by other tools.

\subsubsection{Commands}
\label{sec:ana-sim:scavetool:commands}

\fprog{opp\_scavetool}'s functionality is grouped under four commands:
\ttt{query}, \ttt{export}, \ttt{index}, and \ttt{help}.

\begin{itemize}

\item \tbf{query}: Query the contents of result files. One can list
    runs, run attributes, result items, unique result names, unique module
    names, unique configuration names, etc. One can filter for result types
    (scalar/vector/histogram), and by run, module name, result name and value,
    using match expressions. There are various options controlling the format
    of the output (group-by-runs; grep-friendly; suppress labels; several
    modes for identifying the run in the output, etc.)

\item \tbf{export}: Export results in various formats. Results can be filtered
    by run, module name, result name and more, using match expressions. Output
    vectors can be cropped to a time interval. Several output formats are
    available: CSV in two flavours (one for machine consumption, and a more
    informal one for human consumption via loading into spreadsheet programs),
    {\opp} output scalar/vector file (default), {\opp} SQLite result file, and
    JSON (again two flavours: one strictly adhering to the JSON rules, and
    another one with slightly more relaxed rules but being also more
    expressive). All exporters have multiple options for fine-tuning the output.

\item \tbf{index}: Generate index files (.vci) for vector files. Note that this
    command is usually not needed, as other scavetool commands automatically create
    vector file indices if they are missing or out of date (unless indexing is
    explicitly disabled.) This command can also be used to rebuild a vector file
    so that data are clustered by vectors for more efficient access.

\item \tbf{help}: Prints help. The synopsys is \ttt{opp\_scavetool help <topic>},
    where any command name can be used as topic, plus there are
    additional ones like \ttt{patterns} or \ttt{filters}. \ttt{scavetool
    <command> -h} also works.

\end{itemize}

The default command is \ttt{query}, so its name may be omitted on the
command line.


\subsubsection{Examples}
\label{sec:ana-sim:scavetool:examples}

The following example prints a one-line summary about the contents of
result files in the current directory:

\begin{commandline}
$ opp_scavetool *.sca *.vec
runs: 42   scalars: 294  parameters: 7266  vectors: 22  statistics: 0  ...
\end{commandline}

Listing all results is possible with \fopt{-l}:

\begin{commandline}
$ opp_scavetool -l *.sca *.vec
PureAlohaExperiment-439-20161216-18:56:20-27607:

scalar Aloha.server  duration              26.3156
scalar Aloha.server  collisionLength:mean  0.139814
vector Aloha.host[0] radioState:vector vectorId=2 count=3 mean=0.33 ..
vector Aloha.host[1] radioState:vector vectorId=3 count=9 mean=0.44 ..
vector Aloha.host[2] radioState:vector vectorId=4 count=5 mean=0.44 ..
...
\end{commandline}

To export all scalars in CSV, use the following command:

\begin{commandline}
$  opp_scavetool export -F CSV-R -o x.csv *.sca
Exported 294 scalars, 7266 parameters, 84 histograms
\end{commandline}

The next example writes the queueing and transmission time vectors of
\ttt{sink} modules into a CSV file.

\begin{commandline}
$ opp_scavetool export -f 'module=~**.sink AND ("queueing time" OR "tx time")'
  -o out.csv -F CSV-R *.vec
Exported 15 vectors
\end{commandline}


\section{Result Analysis}
\label{sec:ana-sim:result-analysis}

We recommend the following routes for the analysis of simulation results:

\begin{enumerate}
  \item Use the \textit{Simulation IDE} for casual analysis, i.e. browsing data
        and quick plotting.
  \item Use \textit{Python} (or R) programs for sophisticated analysis and for
        producing tailored reports.
\end{enumerate}

Of courses, many other approaches are also possible, some of which are also
described in later sections. In all cases, \fprog{scavetool} can be used to
filter and export simulation results in a format understood by other tools, for
example CSV.


\subsection{The Analysis Tool in the Simulation IDE}
\label{sec:ana-sim:analysis-tool}

The Simulation IDE provides an Analysis Tool for the analysis and visualization
of simulation results. The Analysis Tool lets you load several result files
at once, and presents their contents somewhat like a database. You can
browse the results, select the particular data you are interested in
(scalars, vectors, histograms), apply processing steps, and display them
in charts or plots. Data selection, processing and charting steps
can be freely combined, resulting in a high degree of freedom.
These steps are grouped into and stored as "recipes", which get automatically
re-applied when new result files are added or existing files are
replaced. This automation spares the user lots of repetitive manual work,
without resorting to scripting.

The Analysis Tool is covered in detail in the User Guide.


\subsection{Spreadsheets}
\label{sec:ana-sim:spreadsheet-programs}
\index{Spreadsheets}

Spreadsheets such as Excel or LibreOffice's Calc, with their charting and
statistical features, offer an obvious alternative to the IDE's Analysis Tool.
Spreadsheets are primarily good for analyzing scalar results, and less so for
vectors and histograms. A commonly available feature called \textit{pivot table}
can be especially useful.

Spreadsheets cannot open result files directly. An obvious solution is to use
\fprog{scavetool} (see \ref{sec:ana-sim:scavetool}) to export data in CSV
(comma-separated vector) format, and open that in the spreadsheet program.
\fprog{scavetool} can also help you merge data from several result
files into a single CSV, and to filter the results to reduce the amount
of data to be loaded into the spreadsheet.

One drawback of using spreadsheet programs is the manual work associated
with preparing and reloading data every time simulations are re-run.


\subsection{Using Python for Result Analysis}
\label{sec:ana-sim:using-python}

Python is a popular general-purpose programming language with a large ecosytem.
It is widely used in the open source community and also commercially in a great
number of application areas. Python is also among the most popular languages
for teaching introductory computer science courses at universities.

In the recent years, Python has emerged as great platform for numerical
computing and statistics due to the appearance of powerful extension packages
like NumPy and Pandas. Some of the Python packages useful for our purposes:

\begin{itemize}
  \item \textit{NumPy} and \textit{SciPy} are numerical and scientific computing
    packages.

  \item \textit{PANDAS} is a data analysis package for Python. The design of
    Pandas has been heavily influenced by R's data frames.

  \item \textit{MatPlotlib} is a plotting library. MatPlotlib provides a
    ``pylab'' API designed to closely resemble that of MATLAB, thereby making
    it easy to learn for experienced MATLAB users.

  \item The \textit{sqlite3} package makes it possible to access SQLite3
    databases from Python. Its significance is that it allows for working
    with {\opp} SQLite result files (see \ref{sec:ana-sim:sqlite-result-files}).

\end{itemize}


\subsection{Using Other Software}
\label{sec:ana-sim:alternative-tools}

\subsubsection{GNU R}
\label{sec:ana-sim:gnu-r}

R is a free software environment for statistical computing and graphics.
R is widely used for statistical software development and data analysis.
The program uses a command line interface, though several graphical user
interfaces are available. It has powerful plotting capabilities,
and it is supported on all major operating systems and platforms.

\begin{hint}
An R package for {\opp} result processing is available from
\url{https://github.com/omnetpp/omnetpp-resultfiles/wiki}.
The package supports loading the contents of {\opp} result files into R,
organizing the data and creating various plots and charts.
The package is well documented, and the web site offers a Tutorial, a Tips
page, a tutorial for the Scalar Lattice GUI package, and other information.
\end{hint}

Some other {\opp}-related packages such as SimProcTC and Syntony also build
on R.

\subsubsection{MATLAB, GNU Octave}
\label{sec:ana-sim:matlab-or-octave}

MATLAB is a commercial numerical computing environment and programming language.
MATLAB allows easy matrix manipulation, plotting of functions and data,
implementation of algorithms, creation of user interfaces, and interfacing
with programs in other languages.

GNU Octave is an MATLAB-like software environment, available on nearly all
platforms. Octave is free software.

\subsubsection{Gnuplot}
\label{sec:ana-sim:gnuplot}

Gnuplot is a simple but popular program for creating two- and three-dimensional
plots of functions and data. The program runs on all major platforms.

Gnuplot has an interactive command interface. For example, given two
data files \texttt{foo.csv} and \texttt{bar.csv} that contain
two values per line ($x$ $y$; such files can be exported with
\fprog{scavetool} from vector files), you can plot them in the same
graph by typing:

\begin{commandline}
plot "foo.csv" with lines, "bar.csv" with lines
\end{commandline}

Several commands are available to adjust the x/y ranges, plotting style, labels,
scaling. The plot can be exported in various image formats.


%%% Local Variables:
%%% mode: latex
%%% TeX-master: "usman"
%%% End:

