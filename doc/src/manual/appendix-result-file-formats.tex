\appendixchapter{Result File Formats}
\label{cha:result-file-formats}

\section{Native Result Files}
\label{cha:result-file-formats:opp}

The file format described here applies to \textit{both output vector and
output scalar files}. Their formats are consistent, only the types of
entries occurring in them are different. This unified format also
means that they can be read with a common routine.

Result files are \textit{line oriented}. A line consists of one or more
tokens, separated by whitespace. Tokens either do not
contain whitespace, or whitespace is escaped using a backslash, or
are quoted using double quotes. Escaping within quotes using
backslashes is also permitted.

The first token of a line usually identifies the type of the entry. A
notable exception is an output vector data line, which begins with a
numeric identifier of the given output vector.

A line starting with \# as the first non{}-whitespace character denotes
a comment, and is to be ignored during processing.

Result files are written from simulation runs. A simulation run
generates physically contiguous sets of lines into one or more result
files. (That is, lines from different runs do not arbitrarily mix in
the files.)

A run is identified by a unique textual \textit{runId}, which appears in
all result files written during that run. The runId may appear on the
user interface, so it should be somewhat meaningful to the user.
Nothing should be assumed about the particular format of runId, but it
will be some string concatenated from the simulated network's name, the
time/date, the hostname, and other pieces of data to make it unique.

A simulation run will typically write into two result files (.vec and
.sca). However, when using parallel distributed simulation, the user
will end up with several .vec and .sca files, because different
partitions (a separate process each) will write into different files.
However, all these files will contain the same runId, so it is possible
to relate data that belong together.

Entry types are:

\begin{itemize}
    \item \textbf{version}: result file version
    \item \textbf{run}: simulation run identifier
    \item \textbf{attr}: run, vector, scalar or statistics object attribute
    \item \textbf{itervar}: iteration variable
    \item \textbf{config}: configuration entry
    \item \textbf{par}: module parameter
    \item \textbf{scalar}: scalar data
    \item \textbf{vector}: vector declaration
    \item \textit{vector-id}: vector data
    \item \textbf{file}: vector file attributes
    \item \textbf{statistic}: statistics object
    \item \textbf{field}: field of a statistics object
    \item \textbf{bin}: histogram bin
\end{itemize}


\subsection{Version}
\label{sec:result-file-formats:opp:version}

Specifies the format of the result file. It is written at the beginning of the file.

Syntax:

\hspace{20mm} \textbf{version} \textit{versionNumber}

The version described in this document is 3, used since {\omnetpp} 6.0.
Version 1 files were produced by {\omnetpp} 3.x and earlier, and version 2
files by {\omnetpp} 4.x and 5.x.\footnote{
    Differences between version 2 and version 3 files are minimal, and mostly only
    affect the run header. Version 3 introduced \textit{itervar} lines to allow
    distinguishing iteration variables from other run attributes (in version 2 they
    were all recorded in \textit{attr} lines). \textit{param} lines in version 2
    (which recorded parameter assignment entries in the configuration) have been
    replaced in version 3 with the more general \textit{config} lines (which record
    all configuration entries, not just parameter assignments). In version 2,
    parameter values (if requested) were recorded as scalars, whereas in version 3
    they are recorded in \textit{par} lines, which allow recording of volatile
    parameters (as expressions) and non-numeric values as well. Additionally,
    version 3 doesn't record the fields \textit{sum} and \textit{sqrsum}
    for weighted statistics.
}



\subsection{Run Declaration}
\label{sec:result-file-formats:opp:run-declaration}

Marks the beginning of a new run in the file. Entries after this line
belong to this run.

Syntax:

\hspace{20mm} \textbf{run} \textit{runId}

Example:

\begin{filelisting}
run TokenRing1-0-20080514-18:19:44-3248
\end{filelisting}

Typically there will be one run per file, but this is not mandatory.
In cases when there are more than one run in a file and it is not feasible
to keep the entire file in memory during analysis, the offsets of the \textit{run}
lines may be indexed for more efficient random access.

The \textit{run} line may be immediately followed by \textit{attribute} lines.
Attributes may store generic data like the network name, date/time of running
the simulation, configuration options that took effect for the simulation, etc.

Run attribute names used by {\opp} include the following:

Generic attribute names:

\begin{itemize}
    \item \textbf{network}: name of the network simulated
    \item \textbf{datetime}: date/time associated with the run
    \item \textbf{processid}: the PID of the simulation process
    \item \textbf{inifile}: the main configuration file
    \item \textbf{configname}: name of the inifile configuration
    \item \textbf{seedset}: index of the seed-set use for the simulation
\end{itemize}

Attributes associated with parameter studies (iterated runs):

\begin{itemize}
    \item \textbf{runnumber}: the run number within the parameter study
    \item \textbf{experiment}: experiment label
    \item \textbf{measurement}: measurement label
    \item \textbf{replication}: replication label
    \item \textbf{repetition}: the loop counter for repetitions with different seeds
    \item \textbf{iterationvars}: string containing the values of the iteration variables
    \item \textbf{iterationvarsf}: like \ttt{iterationvars}, but sanitized for use as part of file names
\end{itemize}

An example run header:

\begin{filelisting}
run PureAlohaExperiment-0-20200304-18:05:49-194559
attr configname PureAlohaExperiment
attr datetime 20200304-18:05:49
attr experiment PureAlohaExperiment
attr inifile omnetpp.ini
attr iterationvars "$numHosts=10, $iaMean=1"
attr measurement "$numHosts=10, $iaMean=1"
attr network Aloha
attr processid 194559
attr repetition 0
attr replication #0
attr resultdir results
attr runnumber 0
attr seedset 0
itervar iaMean 1
itervar numHosts 10
config repeat 2
config sim-time-limit 90min
config network Aloha
config Aloha.numHosts 10
config Aloha.host[*].iaTime exponential(1s)
config Aloha.numHosts 20
config Aloha.txRate 9.6kbps
config **.x "uniform(0m, 1000m)"
config **.y "uniform(0m, 1000m)"
config **.idleAnimationSpeed 1
\end{filelisting}


\subsection{Attributes}
\label{sec:result-file-formats:opp:attributes}

Contains an attribute for the preceding run, vector, scalar or
statistics object. Attributes can be used for saving arbitrary
extra information for objects; processors should ignore unrecognized
attributes.

Syntax:

\hspace{20mm} \textbf{attr} \textit{name} \textit{value}

Example:

\begin{filelisting}
attr network "largeNet"
\end{filelisting}


\subsection{Iteration Variables}
\label{sec:result-file-formats:opp:itervar}

Contains an iteration variable for the given run.

Syntax:

\hspace{20mm} \textbf{itervar} \textit{name} \textit{value}

Examples:

\begin{filelisting}
itervar numHosts 10
itervar tcpType Reno
\end{filelisting}


\subsection{Configuration Entries}
\label{sec:result-file-formats:opp:config}

The configuration of the simulation is captured in the result file as an ordered
list of \textit{config} lines. The list contains both the contents of ini files
and the options given one the command line.

The order of lines represents a flattened view of the ini file(s). The contents
of sections are recorded in an order that reflects the section inheritance
graph: derived sections precede the sections they extend (so \ttt{General} comes
last), and the contents of unrelated sections are omitted. Command like options
are at the top. The relative order of lines within ini file sections are
preserved. This order corresponds to the search order of entries that contain
wildcards (i.e. first match wins).

Values are saved verbatim, except that iteration variables are substituted in
them.

Syntax:

\hspace{20mm} \textbf{config} \textit{key} \textit{value}

Example config lines:

\begin{filelisting}
config sim-time-limit 90min
config network Aloha
config Aloha.numHosts 10
config Aloha.host[*].iaTime exponential(1s)
config **.x "uniform(0m, 1000m)"
\end{filelisting}


\subsection{Scalar Data}
\label{sec:result-file-formats:opp:scalar-data}

Contains an output scalar value.

Syntax:

\hspace{20mm} \textbf{scalar} \textit{moduleName} \textit{scalarName} \textit{value}

Examples:

\begin{filelisting}
scalar "net.switchA.relay" "processed frames" 100
\end{filelisting}

Scalar lines may be immediately followed by \textit{attribute} lines.
{\opp} uses the following attributes for scalars:

\begin{itemize}
    \item \textbf{title}: suggested title on charts
    \item \textbf{unit}: measurement unit, e.g. \ttt{s} for seconds
\end{itemize}


\subsection{Vector Declaration}
\label{sec:result-file-formats:opp:vector-declaration}

Defines an output vector.

Syntax:

\hspace{20mm} \textbf{vector} \textit{vectorId} \textit{moduleName} \textit{vectorName}

\hspace{20mm} \textbf{vector} \textit{vectorId} \textit{moduleName} \textit{vectorName} \textit{columnSpec}

Where \textit{columnSpec} is a string, encoding the meaning and ordering
the columns of data lines. Characters of the string mean:

\begin{itemize}
  \item \textbf{E} event number
  \item \textbf{T} simulation time
  \item \textbf{V} vector value
\end{itemize}

Common values are \ttt{TV} and \ttt{ETV}. The default value is \ttt{TV}.

Vector lines may be immediately followed by \textit{attribute} lines.
{\opp} uses the following attributes for vectors:

\begin{itemize}
    \item \textbf{title}: suggested vector title on charts
    \item \textbf{unit}: measurement unit, e.g. \ttt{s} for seconds
    \item \textbf{enum}: symbolic names for values of the vector;
          syntax is \ttt{"IDLE=0, BUSY=1, OFF=2"}
    \item \textbf{type}: data type, one of \ttt{int}, \ttt{double} and \ttt{enum}
    \item \textbf{interpolationmode}: hint for interpolation mode on the
          chart: \ttt{none} (=do not connect the dots), \ttt{sample-hold},
          \ttt{backward-sample-hold}, \ttt{linear}
    \item \textbf{min}: minimum value
    \item \textbf{max}: maximum value
\end{itemize}


\subsection{Vector Data}
\label{sec:result-file-formats:opp:vector-data}

Adds a value to an output vector. This is the same as in older output
vector files.

Syntax:

\hspace{20mm} \textit{vectorId} \textit{column1} \textit{column2} ...

Simulation times and event numbers \textit{within an output vector} are
required to be in increasing order.

Performance note: Data lines belonging to the same output vector may be
written out in clusters (of size roughly a multiple of the disk's
physical block size). Then, since an output vector file is typically
not kept in memory during analysis, indexing the start offsets of these
clusters allows one to read the file and seek in it more efficiently.
This does not require any change or extension to the file format.

\subsection{Index Header}
\label{sec:result-file-formats:opp:index-header}

The first line of the index file stores the size and modification date
of the vector file. If the attributes of a vector file differ from
the information stored in the index file, then the IDE automatically
rebuilds the index file.

Syntax:

\hspace{20mm} \textbf{file} \textit{filesize} \textit{modificationDate}

\subsection{Index Data}
\label{sec:result-file-formats:opp:index-data}

Stores the location and statistics of blocks in the vector file.

Syntax:

\hspace{20mm} {\textit{vectorId offset length firstEventNo lastEventNo
                       firstSimtime lastSimtime count min max sum sqrsum}}

where

\begin{itemize}
    \item\textit{offset}: the start offset of the block
    \item\textit{length}: the length of the block
    \item\textit{firstEventNo}, \textit{lastEventNo}:
        the event number range of the block (optional)
    \item\textit{firstSimtime}, \textit{lastSimtime}:
        the simtime range of the block
    \item\textit{count, min, max, sum, sqrsum}:
        collected statistics of the values in the block

\end{itemize}

\subsection{Statistics Object}
\label{sec:result-file-formats:opp:statistics-object}

Represents a statistics object.

Syntax:

\hspace{20mm} \textbf{statistic} \textit{moduleName} \textit{statisticName}

Example:

\begin{filelisting}
statistic Aloha.server 	"collision multiplicity"
\end{filelisting}

A \textit{statistic} line may be followed by \textit{field} and \textit{attribute} lines,
and a series of \textit{bin} lines that represent histogram data.

{\opp} uses the following attributes:

\begin{itemize}
    \item \textbf{title}: suggested title on charts
    \item \textbf{unit}: measurement unit, e.g. \ttt{s} for seconds
    \item \textbf{type}: type of the collected values: \ttt{int} or \ttt{double};
                         the default is \ttt{double}
\end{itemize}

A full example with fields, attributes and histogram bins:

\begin{filelisting}
statistic Aloha.server 	"collision multiplicity"
field count 13908
field mean 6.8510209951107
field stddev 5.2385484477843
field sum 95284
field sqrsum 1034434
field min 2
field max 65
attr type int
attr unit packets
bin	-INF	0
bin	0	0
bin	1	0
bin	2	2254
bin	3	2047
bin	4	1586
bin	5	1428
bin	6	1101
bin	7	952
bin	8	785
...
bin	52	2
\end{filelisting}


\subsection{Field}
\label{sec:result-file-formats:opp:field}

Represents a field in a statistics object.

Syntax:

\hspace{20mm} \textbf{field} \textit{fieldName} \textit{value}

Example:

\begin{filelisting}
field sum 95284
\end{filelisting}

Fields:

\begin{itemize}
    \item \tbf{count}: observation count
    \item \tbf{mean}: mean of the observations
    \item \tbf{stddev}: standard deviation
    \item \tbf{min}: minimum of the observations
    \item \tbf{max}: maximum of the observations
    \item \tbf{sum}: sum of the observations
    \item \tbf{sqrsum}: sum of the squared observations
\end{itemize}

For weighted statistics, \textit{sum} and \textit{sqrsum} are replaced
by the following fields:

\begin{itemize}
    \item \tbf{weights}: sum of the weights
    \item \tbf{weightedSum}: the weighted sum of the observations
    \item \tbf{sqrSumWeights}:  sum of the squared weights
    \item \tbf{weightedSqrSum}: weighted sum of the squared observations
\end{itemize}


\subsection{Histogram Bin}
\label{sec:result-file-formats:opp:histogram-bin}

Represents a bin in a histogram object.

Syntax:

\hspace{20mm} \textbf{bin} \textit{binLowerBound} \textit{value}

Histogram name and module is defined on the \textbf{statistic} line,
which is followed by several \textbf{bin} lines to contain data. Any
non{}-\textbf{bin} line marks the end of the histogram data.

The \textit{binLowerBound} column of \textbf{bin} lines represent the
(inclusive) lower bound of the given histogram cell. \textbf{Bin} lines are in
increasing \textit{binLowerBound} order.

The \textit{value} column of a \textbf{bin} line represents the observation
count in the given cell: \textit{value k} is the number of observations
greater or equal to \textit{binLowerBound k}, but smaller than
\textit{binLowerBound k+1}. \textit{Value} is not necessarily an
integer, because the cKSplit and cPSquare algorithms produce
non{}-integer estimates. The first \textbf{bin} line is the underflow
cell, and the last \textbf{bin} line is the overflow cell.

Example:

\begin{filelisting}
bin -INF  0
bin 0 4
bin 2 6
bin 4 2
bin 6 1
\end{filelisting}


\section{SQLite Result Files}
\label{cha:result-file-formats:sqlite}

The database structure in SQLite result files is created with the
following SQL statements. Scalar and vector files are identical in
structure, they only differ in data.

%TODO make an 'sqlite' environment
\begin{filelisting}
CREATE TABLE run
(
    runId       INTEGER PRIMARY KEY AUTOINCREMENT NOT NULL,
    runName     TEXT NOT NULL,
    simtimeExp  INTEGER NOT NULL
);

CREATE TABLE runAttr
(
    runId       INTEGER  NOT NULL REFERENCES run(runId) ON DELETE CASCADE,
    attrName    TEXT NOT NULL,
    attrValue   TEXT NOT NULL
);

CREATE TABLE runItervar
(
    runId       INTEGER  NOT NULL REFERENCES run(runId) ON DELETE CASCADE,
    itervarName TEXT NOT NULL,
    itervarValue TEXT NOT NULL
);

CREATE TABLE runConfig
(
    runId       INTEGER NOT NULL REFERENCES run(runId) ON DELETE CASCADE,
    configKey   TEXT NOT NULL,
    configValue TEXT NOT NULL,
    configOrder INTEGER NOT NULL
);

CREATE TABLE scalar
(
    scalarId      INTEGER PRIMARY KEY AUTOINCREMENT NOT NULL,
    runId         INTEGER  NOT NULL REFERENCES run(runId) ON DELETE CASCADE,
    moduleName    TEXT NOT NULL,
    scalarName    TEXT NOT NULL,
    scalarValue   REAL        -- cannot be NOT NULL, because sqlite converts NaN double value to NULL
);

CREATE TABLE scalarAttr
(
    scalarId      INTEGER  NOT NULL REFERENCES scalar(scalarId) ON DELETE CASCADE,
    attrName      TEXT NOT NULL,
    attrValue     TEXT NOT NULL
);

CREATE TABLE parameter
(
    paramId       INTEGER PRIMARY KEY AUTOINCREMENT NOT NULL,
    runId         INTEGER  NOT NULL REFERENCES run(runId) ON DELETE CASCADE,
    moduleName    TEXT NOT NULL,
    paramName     TEXT NOT NULL,
    paramValue    TEXT NOT NULL
);

CREATE TABLE paramAttr
(
    paramId       INTEGER  NOT NULL REFERENCES parameter(paramId) ON DELETE CASCADE,
    attrName      TEXT NOT NULL,
    attrValue     TEXT NOT NULL
);

CREATE TABLE statistic
(
    statId        INTEGER PRIMARY KEY AUTOINCREMENT NOT NULL,
    runId         INTEGER NOT NULL REFERENCES run(runId) ON DELETE CASCADE,
    moduleName    TEXT NOT NULL,
    statName      TEXT NOT NULL,
    isHistogram   INTEGER NOT NULL,   -- actually, BOOL
    isWeighted    INTEGER NOT NULL,   -- actually, BOOL
    statCount     INTEGER NOT NULL,
    statMean      REAL,   -- note: computed; stored for convenience
    statStddev    REAL,   -- note: computed; stored for convenience
    statSum       REAL,
    statSqrsum    REAL,
    statMin       REAL,
    statMax       REAL,
    statWeights          REAL,  -- note: names of this and subsequent fields are consistent with textual sca file field names
    statWeightedSum      REAL,
    statSqrSumWeights    REAL,
    statWeightedSqrSum   REAL
);

CREATE TABLE statisticAttr
(
    statId        INTEGER NOT NULL REFERENCES statistic(statId) ON DELETE CASCADE,
    attrName      TEXT NOT NULL,
    attrValue     TEXT NOT NULL
);

CREATE TABLE histogramBin
(
    statId        INTEGER NOT NULL REFERENCES statistic(statId) ON DELETE CASCADE,
    lowerEdge     REAL NOT NULL,
    binValue      REAL NOT NULL
);

CREATE TABLE vector
(
    vectorId        INTEGER PRIMARY KEY AUTOINCREMENT NOT NULL,
    runId           INTEGER NOT NULL REFERENCES run(runId) ON DELETE CASCADE,
    moduleName      TEXT NOT NULL,
    vectorName      TEXT NOT NULL,
    vectorCount     INTEGER,    -- cannot be NOT NULL because we fill it in later
    vectorMin       REAL,
    vectorMax       REAL,
    vectorSum       REAL,
    vectorSumSqr    REAL,
    startEventNum   INTEGER,
    endEventNum     INTEGER,
    startSimtimeRaw INTEGER,
    endSimtimeRaw   INTEGER

);

CREATE TABLE vectorAttr
(
    vectorId      INTEGER  NOT NULL REFERENCES vector(vectorId) ON DELETE CASCADE,
    attrName      TEXT NOT NULL,
    attrValue     TEXT NOT NULL
);

CREATE TABLE vectorData
(
    vectorId      INTEGER NOT NULL REFERENCES vector(vectorId) ON DELETE CASCADE,
    eventNumber   INTEGER NOT NULL,
    simtimeRaw    INTEGER NOT NULL,
    value         REAL  -- cannot be NOT NULL because of NaN values
);
\end{filelisting}

Notes:

\begin{enumerate}
  \item To preserve precision, simulation time is stored in raw form, i.e.
        the underlying \ttt{int64} is stored as an integer. To get the real
        value, they have to be multiplied by 10 to the power of the simtime
        exponent, which is global for the simulation run. The simtime
        exponent is stored  in the \ttt{simtimeExp} column of the \ttt{run}
        table.
  \item Some columns like vector statistics are not marked as \ttt{NOT NULL},
        because of technical reasons: their values are not available at the
        time of the insertion, only at the end of the simulation.
  \item \ttt{REAL} columns are not marked as \ttt{NOT NULL}, because
        SQLite stores floating-point NaN values as \ttt{NULL}s.
\end{enumerate}

\begin{caution}
SQLite support in {\opp} is currently experimental, so the above database
structure may change in future releases.
\end{caution}

%%% Local Variables:
%%% mode: latex
%%% TeX-master: "usman"
%%% End:
